\documentclass[9pt]{beamer}

\usepackage{graphicx}
% \usepackage{subfigure}
% \usepackage{array}
\usepackage[latin1]{inputenc}
\usepackage[flushleft]{threeparttable}   
\usepackage{amssymb}
\usepackage{amsfonts}
\usepackage{amsmath}
\usepackage{natbib}
\usepackage{color}
\usepackage{chancery}
\usepackage{tikz}
\usepackage{bm}
\usepackage[gen]{eurosym}
\usepackage{pifont}
\usepackage{fancyvrb}
\usepackage{multirow}
\usepackage{colortbl}
\usepackage{mathptmx}
\usepackage[T1]{fontenc}

% t commands                                        
\newcommand{\firstt}[1]{$\;I(0 \leq t < 5)$}        
\newcommand{\secondt}[1]{$\;I(5 \leq t < 10)$}      
\newcommand{\thirdt}[1]{$\;I(10 \leq t < 15)$}      
\newcommand{\fourtht}[1]{$\;I(15 \leq t < 20)$}     
\newcommand{\fiftht}[1]{$\;I(t \geq 20)$}           

\newcommand{\vect}[1]{\vec{#1}}
\newcommand{\degree}{\ensuremath{^\circ}}
\newcommand{\mix}{\multicolumn{2}{r}{}}       

\usepackage{tikz}
\tikzstyle{mybox} = [very thick, rectangle, inner sep = 10pt]
\usetikzlibrary{arrows,positioning,automata} 

\usepackage{avant}

% structure 
% \usetheme{Frankfurt}
\setbeamertemplate{blocks}[rounded]
\setbeamertemplate{headline}[shadow=false]
\setbeamertemplate{footline}[infolines theme]
\setbeamertemplate{navigation symbols}{}
\setbeamertemplate{itemize items}[triangle]
\setbeamertemplate{enumerate items}[default]

\usecolortheme{beaver}
\definecolor{myred}{RGB}{140, 20, 20} 

% font
\usefonttheme{default}
% \usepackage{fontspec}
% \setsansfont{Verdana}

\title{The Internal Dynamics of the Military and Political Stability}
\subtitle{Social Unrest and Military Coups} 
\author{Gonzalo Rivero}
\institute{YouGov America} 
\date{}

\begin{document} 
	
\begin{frame}[plain]
  \titlepage  
\end{frame}

\section{Introduction}
\subsection{}

\begin{frame}
  \frametitle{Motivating questions} 
  \begin{itemize}
  \item Political stability depends on the ability of the autocrat to keep the
    army at arm's length
  \item Mechanisms for civilian control are not clear in the literature
    \begin{itemize}
    \item Bribes and budget increases
    \item Political concessions
    \item Coup-proofing using paramilitaries
    \end{itemize}
  \item Deciding political make-up of the military is cheaper, safer and more
    efficient.
  \item Any promotion system creates a trade-off between quality and loyalty
  \end{itemize}
  \begin{block}{Main aim}
    Study consequences of political systems in a computational model.
  \end{block}
  
\end{frame}

% \section{Structure}
% \subsection{}
\begin{frame}{Promotions and personnel selection}
  \begin{itemize}
  \item Traditional literature treated the military as a unitary institution in
    which promotions systems would not play a role.
  \item Modern attempts at understanding the effects of factions or coordination
    problems. It brings the problem on the origins of factions and the
    persistence of divisions.
  \item Look at the internal mechanisms that decide the composition of each
    rank, such as promotion rules and appointments. Two extremes:
    \begin{itemize}
    \item Stalin and the purge of 1941: very loyal by inferior army.
    \item Co-opting while appointment the best: very capable but unsympathetic army.
    \end{itemize}
  \item A variety of ways through which the ruler can decide the make-up of the
    military.
  \end{itemize}
\end{frame}

\begin{frame}{Unexplored topic}
  \begin{itemize}
  \item Virtually no literature on promotion rules
  \item Speaks to the literature on the consequences of hierarchical
    institutions, which is very much underdeveloped in political science. 
  \item Scarce research on promotion rules with the exception of studies on the
    Peter effect.
  \end{itemize}
\end{frame}


\begin{frame}
  \frametitle{Simulation}
  \begin{itemize}
  \item The simulation runs a promotion system in an army composed of $N$ soldiers,
    grouped in $G$ units, divided in $K$ ranks. 
  \item Units indicate that a given set of agents that obey to the same
    superior.
  \item Soldiers characterized by ideology and quality in [0, 1]. Model keeps
    tracks of their age and seniority (periods in $k$). 
  \item Soldiers in $k=1$ die randomly (increasing in age). All other ranks die
    with maximum age. 
  \item Replacements in $k>1$ happen via promotions. 
  \item Ruler chooses promotions to general ($k=K$) based on parametric
    combination of ideology and quality.
  \item All other based on proximity to superior in the unit.
  \item Adaptive model allows ruler to set promotion parameters to minimize risk
    (in this case, external)
  \end{itemize}
\end{frame}

\begin{frame}{Evolution of ideology}
  \frametitle{Results}
  All simulations for promotion rules that impose weight $p \in \{0, 0.3, 0.6,
  1\}$ on ideology and $1 - p$ on quality. 
  \begin{figure}[!h]
    \includegraphics[width = 9cm]{../img/ideology_ruler_low_low.png}
    \caption{Evolution of ideology}
  \end{figure}
\end{frame}

\begin{frame}{Evolution of quality}  
  \begin{figure}[!h]
    \centering
    \includegraphics[width = 9cm]{../img/quality_ruler_low_low.png}
    \label{fig:quality_low_low}
    \caption{Evolution of quality (ideology of ruler = 0)}
  \end{figure}
\end{frame}

\begin{frame}{Evolution of ideology}
  \begin{figure}[!h]
  \centering
  \includegraphics[width = 9cm]{../img/ideology_ruler_high_low.png}
  \caption{Evolution of ideology (ideology of ruler = 1)}
\end{figure}
\end{frame}

\begin{frame}{Evolution of quality}
  \begin{figure}[!h]
  \centering
  \caption{Evolution of quality}
  \includegraphics[width = 9cm]{../img/quality_ruler_high_low.png}
    \caption{Evolution of quality (ideology of ruler = 1)}
\end{figure}
\end{frame}

\begin{frame}{Evolution of ideology}
  \begin{figure}[!h]
  \centering
  \includegraphics[width = 9cm]{../img/ideology_ruler_midhigh_midhigh.png}
  \caption{Evolution of ideology (ideology of ruler = 1)}
\end{figure}
\end{frame}

\begin{frame}{Evolution of quality}
  \begin{figure}[!h]
    \centering
  \caption{Evolution of quality}
  \includegraphics[width = 9cm]{../img/quality_ruler_midhigh_midhigh.png}
    \caption{Evolution of quality (ideology of ruler = 1)}
\end{figure}
\end{frame}


\begin{frame}{Conclusions}
  \begin{itemize}
  \item Still preliminary to draw conclusions
  \item Ability to generate factionalization with a parameter change and a
    mechanical rule
  \item Easy to extend to other frameworks such as political parties or firms. 
  \item Interesting application in spatial competition between political
    parties. 
  \end{itemize}
\end{frame}
\end{document}



