\section{Results}
\label{sec:results}


When the ruler is extreme, the variance between units is larger. And more so as
he puts less emphasis on promotion by ideology: as he cares about quality,
variance is larger. 

Difference between ranks increases as he is more radical: in order to get a more
sympathetic command he has to leave lower units behind.

Figure presents the main results of the simulation. It displays the ideological
and the quality composition of the highest rank. It can be see that 

1. It converges quickly
2. The results are intuitive. The composition of the top command converges to
the ideological position of the ruler when the weight on ideology is high. 
3. There is one anomaly, in which the high command is divided into two groups. 


\begin{figure}[!h]
  \centering
  \includegraphics[width=11cm]{../img/evolideology.png}
  \caption{Evolution of quality}
  \label{fig:evolparam}
\end{figure}

Figure takes another look at the anomaly by displaying the average ideology and
quality of each of the ranks. It can be seen that the anomaly coincides with the
point in which all ranks share the same ideology and there is no selection into
higher ranks. 

\begin{figure}[!h]
  \centering
  \includegraphics[width=11cm]{../img/evollower.png}
  \caption{Evolution of lower ranks}
  \label{fig:evolparam}
\end{figure}

Figure explores the anomaly in more detail and how it evolves with the
parameters. 

\begin{figure}[!h]
  \centering
  \includegraphics[width=11cm]{../img/evolparam.png}
  \caption{Evolution of variance}
  \label{fig:evolparam}
\end{figure}

%%% Local Variables:
%%% mode: latex
%%% TeX-master: "promotions_paper"
%%% End:
