\section{Results}
\label{sec:results}

The next two figures (figures \ref{fig:ideology_low_low} and
\ref{fig:quality_low_low}) show the evolution of the ideology and the quality
for the officers of rank $K$ in a simulation in which the ideology of the ruler
is set to 0 and the weight on the overall quality of the military is also set to
0, meaning that the ruler does not pay attention to the risk associated with a
given composition of the military when choosing a promotion rule. In the
simulations on the left, the parameters used in the promotion to rank $K$ are
set so that the weight in ideology is $p$ and the weight on quality is $(1-p)$
for $p \in \{0, 0.3, 0.6, 1\}$ (top to bottom figure), and the weight on
seniority is always 0. In the simulations on the right, the ruler changes those
parameters freely with the aim of minimizing the overall risk he faces. The same
structure is used for all the other figures.

The results on the left are what we would have expected. The ruler manages to
get a high command in which all its members are from an ideology close to his
(figure~\ref{fig:ideology_low_low}) and yet he is still able to select from a
pool of high quality candidates (figure \ref{fig:quality_low_low}) regardless of
the parameters that the promotion rule takes. The results on the left, although
using the same parameters, converge to different results, which point to two
different attractors.

\begin{figure}[!h]
  \centering
  \caption{Evolution of ideology}
  \includegraphics[width = 14cm]{../img/ideology_ruler_low_low.png}
  \label{fig:ideology_low_low}
\end{figure}

\begin{figure}[!h]
  \centering
  \caption{Evolution of quality}
  \includegraphics[width = 14cm]{../img/quality_ruler_low_low.png}
  \label{fig:quality_low_low}
\end{figure}

\clearpage

In figures \ref{fig:ideology_low_high} and \ref{fig:quality_low_high}, the ruler
still has an ideology of 0 but now puts full weight on the quality of the
military. The results of the constant model reproduce what was observed before.
The search process produces again the two attractors, and interestingly, it
converges to them even when it has explored different combinations of ideology
and quality (top right panel of figures \ref{fig:ideology_low_high} and
\ref{fig:quality_low_high}).
 
\begin{figure}[!h]
  \centering
  \caption{Evolution of ideology}
  \includegraphics[width = 14cm]{../img/ideology_ruler_low_high.png}
  \label{fig:ideology_low_high}
\end{figure}

\begin{figure}[!h]
  \centering
  \caption{Evolution of quality}
  \includegraphics[width = 14cm]{../img/quality_ruler_low_high.png}
  \label{fig:quality_low_high}
\end{figure}

\clearpage 

If we switch the ideology of the ruler to a value 1 and repeat the same
exercises as above (figures \ref{fig:ideology_high_low} and
\ref{fig:quality_high_low}), we see two things happening. First, that the
constant model switches attractor after the parameter for ideology goes above
0.5 (compare the second and third panels on the left of the two figures). There
is therefore some asymmetry in the model that makes the value of the promotion
rule parameters affect more clearly and have more impact when the ideology of
the ruler is 1 than when it is 0. The other interesting effect, is that the
adaptive model converges to a situation in which its distribution has greater
variance and it is centered around 0.5.

\begin{figure}[!h]
  \centering
  \caption{Evolution of ideology}
  \includegraphics[width = 14cm]{../img/ideology_ruler_high_low.png}
  \label{fig:ideology_high_low}
\end{figure}

\begin{figure}[!h]
  \centering
  \caption{Evolution of quality}
  \includegraphics[width = 14cm]{../img/quality_ruler_high_low.png}
  \label{fig:quality_high_low}
\end{figure}

\clearpage 

However, the most interesting results arise in intermediate scenarios when the
ideology of the ruler and the weight he poses on the quality of the military are
strictly between 0 and 1. In this case, a constant rule produces a clear
factionalization of the military (second panel in figures
\ref{fig:quality_midhigh_midhigh} and \ref{fig:quality_midhigh_midhigh}), which
seems to be common for all cases in which the weight on quality is 0.7
regardless of the other parameters. The adaptive model only partially reproduces
this behavior but the factionalization result is not sustain forever. 

\begin{figure}[!h]
  \centering
  \caption{Evolution of ideology}
  \includegraphics[width = 14cm]{../img/ideology_ruler_midhigh_midhigh.png}
  \label{fig:quality_midhigh_midhigh}
\end{figure}

\begin{figure}[!h]
  \centering
  \caption{Evolution of quality}
  \includegraphics[width = 14cm]{../img/quality_ruler_midhigh_midhigh.png}
  \label{fig:quality_midhigh_midhigh}
\end{figure}

\clearpage

%%% Local Variables:
%%% mode: latex
%%% TeX-master: "promotions_paper"
%%% End:
