\section{Simulation}
\label{sec:simulation}

Army composed of $N$ soldiers, grouped in $G$ units, and divided in $K$ ranks.

A soldier is characterized by three exogenous attributes, namely an ideological position, a quality, and age, and an
endogenously determined rank.

More formaly, a soldier $i$ is characterized by a vector $(\theta_i, \mu_i)$, where $\theta_i \sim U(0,1)$, $\mu_i \sim U(-1,1)$, $a_i
\in \{1, \dots ,K\}$, $r_i \in \{1, \dots, K\}$

The simulation analyzes the behavior of three promotion rules, with two different variations. 

Let $j$ be a vacant at rank $r$, and let $I = \{i \in I | r_i < r_j \text{ and } a_i < \overline{a}\}$ be the set of
vacants. Promotion at random picks 
\begin{eqnarray*}
 \Pr(i \in I) = \Pr(j \in I), i \neq j
\end{eqnarray*}

Promotion of the ablest picks 
\begin{eqnarray*}
  \arg\max_{i \in I} \mu_i
\end{eqnarray*}

while promotion of the closests picks
\begin{eqnarray*}
  \arg\min_{i \in I} \abs{\theta_i - \hat \theta}
\end{eqnarray*}

I have also explored the behavior of restrictions $I = \{i \in I | r_i = r_j - 1 \text{ and } a_i <
\overline{a}\}$. This set reduces the possible promotions to those individuals in the rank inmediately below the
available position.

Define $c_i^j = \{0, 1\}$ to be such that $i$ has command over $j$. Then, for an individual $i$ of rank $r$ define the
set $C_i = \{j \in N | r_j = r_i - 1 \text{ and } c_i^j = 1\}$. Recursively, it defines for an individual $i$ the set of
soldiers over whom he has direct or indirect command.

In the simulations that follow I have used the following parameters: 

\begin{table}[!h]
  \centering
  \caption{Parameters used in the simulation}
  \begin{tabular}{ll}
    \hline
    N & 1024 \\
    Ranks & 4 \\ 
    U & 4 \\
    Age & 10 \\
    \hline
  \end{tabular}
  \label{tab:parameters}
\end{table}
Python script. Code available at the following \href{https://griverorz@bitbucket.org/griverorz/promotions}{Mercurial repository}

\section{Results}
\label{sec:results}

Figure \ref{fig:ideology} shows the evolution of the mean of the ideological position of the military divided by rank
and promotion system. Not surprisingly, promoting those that are closer to the ruler makes the highest ranks very
homogeneous. In fact, ranks are closer to the ruler the higher their position in the hierarchy is, even although this
configuration is achieved by actually moving the lower ranks \emph{further apart}. The rest of the promotion rules
maintain an ideologically neutral army, at least if we define it from a similar ideological distribution across
ranks. In particular, and as we would expect, each rank looks like composed by random extractions of ideology.

Pomotion with seniority shows a lower oscillation.

\begin{figure}[!h]
  \centering
  \caption{Evolution of ideology}
  \includegraphics[width = 14cm]{figures/ideology.pdf}
  \label{fig:ideology}
\end{figure}

The natural question to ask for a promotion system is how it affects the overall quality of the organization with
respect to the task it is supposed to perform. From a very intuitive perspective, the criterion for order in the efficiency two
different armies is by comparing their resulting sorting. In a highly hierarchized
institution like a military in which the degree of substituibility of a given element decreases with the rank ---the
tasks they perform are more specialized and entail more responsibility---, it is natural to require the individuals with
the highest quality are concentrated in the highest positions. I have then calculated the total efficiency of the
institution as the weighted sum of the individual quality, where weights are given by the rank of the
individual. For simplicity, I have normalized this value with respect to the maximum achievable efficiency, as given by
a military in which all individuals are endowed with the highest possible value of quality --- figure \ref{fig:efficiency}.

\begin{figure}[!h]
  \centering
  \caption{Evolution of efficiency}
  \includegraphics[width = 14cm]{figures/efficiency.pdf}
  \label{fig:efficiency}
\end{figure}

In terms of risk, not individuals should carry the same weight. If we assume that some degree of vertical obedience will
exist, it is then fit to assume that rebellions by highest ranks will entail higher risk for the ruler than threats from
lower ranks. I have captured this idea in the simulated armies by looking at the utility of each individual weighted by
the amount of soldier-quality in can mobilized. Figure \ref{fig:wutility} shows the utility level as an approximation of
risk weighted by the total quality that each individual can, directly or indirectly, mobilize against the ruler.

% % \begin{sideways}
%   \begin{figure}[!h]
%     \centering
%     \caption{Weighted utility}
%     \includegraphics[width = 14cm]{figures/weighted_utility.png}
%     \label{fig:wutility}
%   \end{figure}
% % \end{sideways}

  % \begin{figure}[!h]
  %   \centering
  %   \caption{Evolution of utility}
  %   \includegraphics[width = 14cm]{figures/utility.pdf}
  %   \label{fig:utility}
  % \end{figure}

% The individual trajectory of an individual is shown in figure \ref{fig:evolution}.

  \begin{figure}[!h]
    \centering
    \caption{Evolution of individual risk}
    \includegraphics[width = 14cm]{figures/capacity.pdf}
    \label{fig:evolution}
  \end{figure}

%%% Local Variables: 
%%% mode: latex
%%% TeX-master: "promotions_paper"
%%% End: 
