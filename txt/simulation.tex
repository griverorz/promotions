\section{Simulation}
\label{sec:simulation}

The simulation runs a promotion system in an army composed of $N$ soldiers,
grouped in $G$ units, divided in $K$ ranks. Units indicate that a given set of
agents that obey to the same superior.

Each soldier is characterized by three exogenous attributes, namely an
ideological position in $x\in[0, 1]$, quality in $q\in[0, 1]$, and age
$a=1, \dots, A$, and an endogenously determined rank from $k=1, \dots, K$, where
$K$ denotes the highest rank. At each iteration $t =1, \dots, T$, the model
updates the \emph{seniority} of the agent at a given rank, which is the number
of periods the agent has remained at rank $k$. The seniority is restarted every
time the agent is promoted to a new position and it is increased by one every
period the agent remains in the same position.

Agents survive in the model for a certain number of periods given by $A$. Once
an agent reaches the maximum age, it deterministically dies and its position is
open for replacement. Therefore, in the simulation, the death of old officers is
the only mechanism for creating available slots in the hierarchy ---with the
exception of the lowest rank, $k=1$. In order to limit the median age of the
structure, the simulation at every period $t$ removes agents ---from the lowest
rank only--- at random agents that have not reached the maximum age with
probability $a_i/A$. New soldiers are then created to replace the old ones with
the values of age, ideology, and quality given by:
\begin{eqnarray}
  & \frac{a_i}{(A - 1) + 1}  \sim \text{Beta}((K + 1) - k) \\
  & (q_i, x_i) \sim \text{Dirichlet}(2, 2) 
\end{eqnarray}
where $k=1$. The same process is used for generating the army at the intial
period, using a general $k$.

Therefore, at each $t$, the army creates two types of vacancies. Vacancies at
the lowest rank are filled randomly using the process described above. Vacancies
for all other ranks are filled using promotions from the immediately lower
ranks. The promotion system is entirely determined by the immediate superior to
the vacant position (the \emph{picker}). In order to simplify the structure of
decisions, for all vacancies except the highest possible rank, the picker
chooses, from the pool of available candidates the individual with value of
ideology closest to his. In the simulations that follow, the pool of candidates
for a position is considered to be the soldiers that belong to the unit of the
vacancy who are in the immediately next rank.

For the case of highest rank, the \emph{generals}, the picker is set to be an
external agent, the \emph{ruler}, who is endowed with an ideology and a
promotion rule. From among the possible candidates, the ruler picks the one that
maximizes the individual score, which is a weighted combination of the
ideological distance between candidate and ruler, the seniority, and the quality
of the candidate.

In the remainder of the paper, I perform two types of simulations. In the first
one, the parameters for each dimension of ideology, seniority, and quality are
taken as fixed for all iterations. In the second one, the parameters are set by
the ruler using an adaptive rule in order to maximize the overall quality of the
military, which is the average quality of the military weighted by their
seniority and rank.This value can be interpreted as an inverse measure of the
risk to the ruler in case of a challenge. Therefore, in each iteration, the
ruler adjusts the components of the promotion rule in order to minimize risk.
The adaptation rule is a simple search strategy: if the risk between $t$ and
$t-1$ increases, the ruler will change direction in the parameter space.
Otherwise, he will keep moving the parameters in the same direction.

Therefore, the simulation emphasizes a military that is decentralized to some
degree in the sense that each unit performs promotions based on their own
personal preferences. This idea reflects the lack of observability by the ruler
of the actions of the different units. At the same time, in this simulation,
only the ruler is concerned with the overall quality of the military.

The code for the simulation is available at the following
\href{https://griverorz@github.org/griverorz/promotions}{GitHub repository}

%%% Local Variables:
%%% mode: latex
%%% TeX-master: "promotions_paper"
%%% End:
