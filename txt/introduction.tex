\section{Introduction}

The mechanisms for achieving civilian control over the military are still
unclear. Political concessions \citep{leon:2009,svolik:2012a}, bribes
\citep{goda:2000}, increases in military spending \citep{rivero:2011}, purges,
\dots{} they all useful tools for the ability of the autocrat to co-opt and
coerce \citep{gandhi:2007} the military command into obedience ---which usually
amounts to repressing the enemies of the regime when they hit the street while
not rebelling even when the chances of succeeding are high.

Political stability hinges on the ability of the autocrat to keep the army at an
arm's length. We know that autocrats are more likely to be deposed than to
retire in office \citep{escriba:2013}, and we also have evidence showing that
coups are more of a threat to the regime than popular uprisings
\citep{gandhi:2008}. 

But the literature has mostly explored mechanisms of civilian control that take
the composition of the military as given \citep{acemoglu:2010}. And yet, it
seems reasonable that the autocrat may be able to establish control over the top
command by appointing and removing officials or, more generally, by deciding
over the makeup of the military through a manipulation of the promotion rules.
At the end of the day, any mechanism to ensure obedience will be more effective
when the top brass is already loyal or sympathetic.

However, a selection strategy of the military command creates a clear trade-off.
If the autocrat promotes officers based on political sympathies alone, he risks
reducing the overall effectiveness of the military. In this case, the question
is how can a dictator survive against its internal and external enemies with a
military composed by loyal but dim-witted soldiers? The other extreme, favoring
only high-quality candidates, leaves him in no better position. Promoting the
best and brightest will certainly lead to a situation in which the ruler will
not have the personal loyalty of his command. In that case, obedience will
either come at a high price ---measured in either economic transfers or
political concessions--- or will not come at all ---if the distance between the
ruler and the command is large enough to defeat any compensation that the rule
may offer in exchange for blind obedience. It seems apparent that the ruler must
trade quality for political sympathy.

In this paper, I study the trade-offs between the determinants of promotions
systems using a adaptive agent-based model in which a ruler chooses over the
promotion method to be implemented as he faces internal and external
constraints.

%%% Local Variables:
%%% mode: latex
%%% TeX-master: "promotions_paper"
%%% End:
