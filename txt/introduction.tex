\section{Introduction}

In spite of their presence everywhere, hierarchical institution remain one of
the least understood objects in political science. 

Most arguments about political stability in dictatorships hinge on the ability
of the autocrat to keep the army at an arm's length. However, the mechanisms for
achieving this situation are never clearly specified. 

If the autocrat promotes based on political sympathies alone, he risks reducing
the overall effectiveness of the military that he would have achieved by
promoting on professional merits alone. In this case, the question is how can a
dictator survive with loyal but cazurros soldiers?

The other situation leaves him in no better position. Promoting the best and
brightest will certainly lead to a situation in which the ruler does not have
the personal loyalty of his command.

Besides, there are other constraints that constraint the decision of the
dictator. There are elemental criteria that he can or should not ignore: even in
the most tightly controlled militaries, promotions are awarded based on the
escalafon and some degree of seniority is expected in each position.

I analyze here the trade-offs between the determinants of promotions systems
using a adaptive agent-based model in which a ruler chooses over the promotion
method to be implemented as he faces internal and external constraints. 

Results


%%% Local Variables:
%%% mode: latex
%%% TeX-master: "promotions_paper"
%%% End:
