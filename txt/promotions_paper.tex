\documentclass[11pt]{article}
\linespread{1.1}

%%%%%%%%%%%%%%%%%%%%%%%%%%%%%%%%%%%%%%%%%%%%%%%%%%%%%%%%%%
\usepackage[pdftex]{graphicx}
\usepackage[utf8]{inputenc} 
\usepackage{amssymb} 
% \usepackage{mla}
\usepackage{amsmath} 
\usepackage{tikz}
\usepackage{bbding}
\usepackage{booktabs,caption,fixltx2e}
\usepackage[flushleft]{threeparttable}
\usepackage{array}
\usepackage{amsfonts} 
\usepackage{listings} 
\usepackage{bm}
\usepackage{pdflscape}
\usepackage{setspace} 
\usepackage{rotating}
\usepackage{pifont}
\usepackage{multirow}
\usepackage{enumerate}
\usepackage{makeidx} 
\usepackage{alltt} 
\usepackage{fancyvrb} 
\usepackage{soul}
\usepackage{scalefnt} 
\usepackage{calc} 
\usepackage[toc,page]{appendix}
\usepackage{hyperref} 
\usepackage{natbib} 
\usepackage{fancyhdr} 
\usepackage{pdfpages}
\usepackage{framed}
\usepackage{mdwlist}
\usepackage{geometry}
% \geometry{left = 2.5cm, right = 2.5cm, top = 2.5cm, bottom = 3cm}
\usepackage[T1]{fontenc}
% \usepackage[osf]{mathpazo}
\usepackage{titlesec}

\renewcommand*{\figureautorefname}{figure}
\renewcommand*{\figureautorefname}{table}

\tikzstyle{mybox} = [very thick, rectangle, inner sep = 10pt]
\usetikzlibrary{arrows,positioning,automata} 

\newcommand{\bb}[1]{\multicolumn{6}{c}{#1}}
\newcommand{\mix}{\multicolumn{2}{r}{}}
\newcommand{\vect}[1]{\boldsymbol{#1}}
\newcommand{\degree}{\ensuremath{^\circ}}
\newcommand{\E}{\ensuremath{\mathbb{E}}}

% Technical environments
\newtheorem{theorem}{Theorem}
\newtheorem{lemma}{Result}
\newtheorem{assumption}{Assumption}
\newtheorem{proposition}[theorem]{Proposition}
\newtheorem{corollary}[theorem]{Corollary}

\newenvironment{proof}[1][Proof]{\begin{trivlist}
\item[\hskip \labelsep {\bfseries #1}]}{\end{trivlist}}
\newenvironment{definition}[1][Definition]{\begin{trivlist}
\item[\hskip \labelsep {\bfseries #1}]}{\end{trivlist}}
\newenvironment{example}[1][Example]{\begin{trivlist}
\item[\hskip \labelsep {\bfseries #1}]}{\end{trivlist}}
\newenvironment{remark}[1][Remark]{\begin{trivlist}
\item[\hskip \labelsep {\bfseries #1}]}{\end{trivlist}}

\newcommand{\qed}{\nobreak \ifvmode \relax \else
      \ifdim\lastskip<1.5em \hskip-\lastskip
      \hskip1.5em plus0em minus0.5em \fi \nobreak
      \vrule height0.75em width0.5em depth0.25em\fi}

% New operator
\DeclareMathOperator{\supp}{supp}

% Spaces in bibliography
\let\oldthebibliography=\thebibliography
\let\endoldthebibliography=\endthebibliography
\renewenvironment{thebibliography}[1]{%
  \begin{oldthebibliography}{#1}%
    \setlength{\parskip}{0ex}%
    \setlength{\itemsep}{0.1ex}%
  }%
  {%
  \end{oldthebibliography}%
}

\clubpenalty=10000  
\widowpenalty=10000

\renewcommand{\tablename}{\bf Table}
\renewcommand{\thetable}{\Roman{table}}
\renewcommand{\figurename}{\bf Figure}

\definecolor{lightred}{rgb}{0.7,0,0}
\definecolor{darkgreen}{rgb}{0,0.8,0}
\definecolor{lightblue}{rgb}{0,0,0.7}

\makeatletter
% \renewcommand\@seccntformat[1]{\csname \the#1. \endcsname}

% \sectionfont{verdana}
\renewcommand\section{\@startsection {section}{1}{\z@}%
  {-3.5ex \@plus -1ex \@minus -.2ex}%
  {2.3ex \@plus.2ex}%
  {\color{lightblue}\centering\Large\itshape}}

\renewcommand\subsection{\@startsection {subsection}{2}%
  {0mm}%
  {-\baselineskip}%
  {0.5\baselineskip}%
  {\color{lightblue}\centering\large\itshape}}
\makeatother

% \newcommand{\ssection}[1]{\section[#1]{\centering #1}}

\hypersetup{colorlinks,
  linkcolor = lightred,
  filecolor = lightred,
  urlcolor = lightblue,
  citecolor = lightred}

\providecommand{\abs}[1]{\lvert#1\rvert}
\providecommand{\norm}[1]{\lVert#1\rVert}
\providecommand{\up}[1]{\overline{#1}}
\providecommand{\lo}[1]{\underline{#1}}

% Dutch style of paragraph formatting
\parskip 5pt

% Separation between lines
\doublerulesep 1pt

\begin{document}

\title{A simple mechanism of factionalization in hierarchical organizations}

\author{\href{mailto:griverorz@gmail.com}{Gonzalo Rivero}\footnote{Contact:
    \href{mailto:griverorz@gmail.com}{griverorz@gmail.com}.} \\ YouGov America }

\date{\today \\ \vspace*{2cm}  WORK IN PROGRESS --- COMMENTS ARE WELCOME.}

\maketitle


\begin{abstract}
  In this paper I explore the consequences of different promotion systems using
  an agent-based model. The trade-off between a competent and a sympathetic
  military that an autocrat faces when deciding who to appoint to senior
  positions in the military is explicitly modeled in an adaptive framework.
  Results suggest a mechanic effect of the promotion system as a possible source
  of factionalization within the military.
\end{abstract}

\newpage

\section{Introduction}

Most arguments about political stability in dictatorships hinge on the ability
of the autocrat to keep the army at an arm's length. However, the mechanisms for
achieving this situation are never clearly specified. 

If the autocrat promotes based on political sympathies alone, he risks reducing
the overall effectiveness of the military that he would have achieved by
promoting on professional merits alone. In this case, the question is how can a
dictator survive with loyal but cazurros soldiers?

The other situation leaves him in no better position. Promoting the best and
brightest will certainly lead to a situation in which the ruler does not have
the personal loyalty of his command.

Besides, there are other constraints that constraint the decision of the
dictator. There are elemental criteria that he can or should not ignore: even in
the most tightly controlled militaries, promotions are awarded based on the
escalafon and some degree of seniority is expected in each position.

I analyze here the trade-offs between the determinants of promotions systems
using a adaptive agent-based model in which a ruler chooses over the promotion
method to be implemented as he faces internal and external constraints. 

Results


%%% Local Variables:
%%% mode: latex
%%% TeX-master: "promotions_paper"
%%% End:


\section{Theory}

Sometimes one skips seniority and the ladder altogether. For instance, X
promoted the members of his cabinet to captains. Similarly, X skipped few ranks.
However, those situations are rare and sometimes lead to violent reactions from
the rest of the military. For instance, when the newly appointed dictator
decided to promote himself to general, a coup ensued. Much more common are
situations in which the ruler bends the rules slightly. For instance, it was
common knowledge that (scretario de franco) skipped the mandatory instruction
for his grade of Admiral. 

Franco granted general rank or higher to people wanted to put out of circulation.

We know of many situations in which the autocrat puts too much emphasis
on one dimension. But some of those end up in either a violent removal through a
civil war (CAR or Sierra Leone) or a straight removal via a coup.

\section{Simulation}
\label{sec:simulation}

Army composed of $N$ soldiers, grouped in $G$ units, and divided in $K$ ranks.

A soldier is characterized by three exogenous attributes, namely an ideological position, a quality, and age, and an
endogenously determined rank.

More formaly, a soldier $i$ is characterized by a vector $(\theta_i, \mu_i)$, where $\theta_i \sim U(0,1)$, $\mu_i \sim U(-1,1)$, $a_i
\in \{1, \dots ,K\}$, $r_i \in \{1, \dots, K\}$

The simulation analyzes the behavior of three promotion rules, with two different variations. 

Let $j$ be a vacant at rank $r$, and let $I = \{i \in I | r_i < r_j \text{ and } a_i < \overline{a}\}$ be the set of
vacants. Promotion at random picks 
\begin{eqnarray*}
 \Pr(i \in I) = \Pr(j \in I), i \neq j
\end{eqnarray*}

Promotion of the ablest picks 
\begin{eqnarray*}
  \arg\max_{i \in I} \mu_i
\end{eqnarray*}

while promotion of the closests picks
\begin{eqnarray*}
  \arg\min_{i \in I} \abs{\theta_i - \hat \theta}
\end{eqnarray*}

I have also explored the behavior of restrictions $I = \{i \in I | r_i = r_j - 1 \text{ and } a_i <
\overline{a}\}$. This set reduces the possible promotions to those individuals in the rank inmediately below the
available position.

Define $c_i^j = \{0, 1\}$ to be such that $i$ has command over $j$. Then, for an individual $i$ of rank $r$ define the
set $C_i = \{j \in N | r_j = r_i - 1 \text{ and } c_i^j = 1\}$. Recursively, it defines for an individual $i$ the set of
soldiers over whom he has direct or indirect command.

In the simulations that follow I have used the following parameters: 

\begin{table}[!h]
  \centering
  \caption{Parameters used in the simulation}
  \begin{tabular}{ll}
    \hline
    N & 1024 \\
    Ranks & 4 \\ 
    U & 4 \\
    Age & 10 \\
    \hline
  \end{tabular}
  \label{tab:parameters}
\end{table}
Python script. Code available at the following \href{https://griverorz@bitbucket.org/griverorz/promotions}{Mercurial repository}

\section{Results}
\label{sec:results}

Figure \ref{fig:ideology} shows the evolution of the mean of the ideological position of the military divided by rank
and promotion system. Not surprisingly, promoting those that are closer to the ruler makes the highest ranks very
homogeneous. In fact, ranks are closer to the ruler the higher their position in the hierarchy is, even although this
configuration is achieved by actually moving the lower ranks \emph{further apart}. The rest of the promotion rules
maintain an ideologically neutral army, at least if we define it from a similar ideological distribution across
ranks. In particular, and as we would expect, each rank looks like composed by random extractions of ideology.

Pomotion with seniority shows a lower oscillation.

\begin{figure}[!h]
  \centering
  \caption{Evolution of ideology}
  \includegraphics[width = 14cm]{figures/ideology.pdf}
  \label{fig:ideology}
\end{figure}

The natural question to ask for a promotion system is how it affects the overall quality of the organization with
respect to the task it is supposed to perform. From a very intuitive perspective, the criterion for order in the efficiency two
different armies is by comparing their resulting sorting. In a highly hierarchized
institution like a military in which the degree of substituibility of a given element decreases with the rank ---the
tasks they perform are more specialized and entail more responsibility---, it is natural to require the individuals with
the highest quality are concentrated in the highest positions. I have then calculated the total efficiency of the
institution as the weighted sum of the individual quality, where weights are given by the rank of the
individual. For simplicity, I have normalized this value with respect to the maximum achievable efficiency, as given by
a military in which all individuals are endowed with the highest possible value of quality --- figure \ref{fig:efficiency}.

\begin{figure}[!h]
  \centering
  \caption{Evolution of efficiency}
  \includegraphics[width = 14cm]{figures/efficiency.pdf}
  \label{fig:efficiency}
\end{figure}

In terms of risk, not individuals should carry the same weight. If we assume that some degree of vertical obedience will
exist, it is then fit to assume that rebellions by highest ranks will entail higher risk for the ruler than threats from
lower ranks. I have captured this idea in the simulated armies by looking at the utility of each individual weighted by
the amount of soldier-quality in can mobilized. Figure \ref{fig:wutility} shows the utility level as an approximation of
risk weighted by the total quality that each individual can, directly or indirectly, mobilize against the ruler.

% % \begin{sideways}
%   \begin{figure}[!h]
%     \centering
%     \caption{Weighted utility}
%     \includegraphics[width = 14cm]{figures/weighted_utility.png}
%     \label{fig:wutility}
%   \end{figure}
% % \end{sideways}

  % \begin{figure}[!h]
  %   \centering
  %   \caption{Evolution of utility}
  %   \includegraphics[width = 14cm]{figures/utility.pdf}
  %   \label{fig:utility}
  % \end{figure}

% The individual trajectory of an individual is shown in figure \ref{fig:evolution}.

  \begin{figure}[!h]
    \centering
    \caption{Evolution of individual risk}
    \includegraphics[width = 14cm]{figures/capacity.pdf}
    \label{fig:evolution}
  \end{figure}

%%% Local Variables: 
%%% mode: latex
%%% TeX-master: "promotions_paper"
%%% End: 

\section{Results}
\label{sec:results}

The next two figures (figures \ref{fig:ideology_low_low} and
\ref{fig:quality_low_low}) show the evolution of the ideology and the quality
for the officers of rank $K$ in a simulation in which the ideology of the ruler
is set to 0 and the weight on the overall quality of the military is also set to
0, meaning that the ruler does not pay attention to the risk associated with a
given composition of the military when choosing a promotion rule. In the
simulations on the left, the parameters used in the promotion to rank $K$ are
set so that the weight in ideology is $p$ and the weight on quality is $(1-p)$
for $p \in \{0, 0.3, 0.6, 1\}$ (top to bottom figure), and the weight on
seniority is always 0. In the simulations on the right, the ruler changes those
parameters freely with the aim of minimizing the overall risk he faces. The same
structure is used for all the other figures.

The results on the left are what we would have expected. The ruler manages to
get a high command in which all its members are from an ideology close to his
(figure~\ref{fig:ideology_low_low}) and yet he is still able to select from a
pool of high quality candidates (figure \ref{fig:quality_low_low}) regardless of
the parameters that the promotion rule takes. The results on the left, although
using the same parameters, converge to different results, which point to two
different attractors.

\begin{figure}[!h]
  \centering
  \caption{Evolution of ideology}
  \includegraphics[width = 14cm]{../img/ideology_ruler_low_low.png}
  \label{fig:ideology_low_low}
\end{figure}

\begin{figure}[!h]
  \centering
  \caption{Evolution of quality}
  \includegraphics[width = 14cm]{../img/quality_ruler_low_low.png}
  \label{fig:quality_low_low}
\end{figure}

\clearpage

In figures \ref{fig:ideology_low_high} and \ref{fig:quality_low_high}, the ruler
still has an ideology of 0 but now puts full weight on the quality of the
military. The results of the constant model reproduce what was observed before.
The search process produces again the two attractors, and interestingly, it
converges to them even when it has explored different combinations of ideology
and quality (top right panel of figures \ref{fig:ideology_low_high} and
\ref{fig:quality_low_high}).
 
\begin{figure}[!h]
  \centering
  \caption{Evolution of ideology}
  \includegraphics[width = 14cm]{../img/ideology_ruler_low_high.png}
  \label{fig:ideology_low_high}
\end{figure}

\begin{figure}[!h]
  \centering
  \caption{Evolution of quality}
  \includegraphics[width = 14cm]{../img/quality_ruler_low_high.png}
  \label{fig:quality_low_high}
\end{figure}

\clearpage 

If we switch the ideology of the ruler to a value 1 and repeat the same
exercises as above (figures \ref{fig:ideology_high_low} and
\ref{fig:quality_high_low}), we see two things happening. First, that the
constant model switches attractor after the parameter for ideology goes above
0.5 (compare the second and third panels on the left of the two figures). There
is therefore some asymmetry in the model that makes the value of the promotion
rule parameters affect more clearly and have more impact when the ideology of
the ruler is 1 than when it is 0. The other interesting effect, is that the
adaptive model converges to a situation in which its distribution has greater
variance and it is centered around 0.5.

\begin{figure}[!h]
  \centering
  \caption{Evolution of ideology}
  \includegraphics[width = 14cm]{../img/ideology_ruler_high_low.png}
  \label{fig:ideology_high_low}
\end{figure}

\begin{figure}[!h]
  \centering
  \caption{Evolution of quality}
  \includegraphics[width = 14cm]{../img/quality_ruler_high_low.png}
  \label{fig:quality_high_low}
\end{figure}

\clearpage 

However, the most interesting results arise in intermediate scenarios when the
ideology of the ruler and the weight he poses on the quality of the military are
strictly between 0 and 1. In this case, a constant rule produces a clear
factionalization of the military (second panel in figures
\ref{fig:quality_midhigh_midhigh} and \ref{fig:quality_midhigh_midhigh}), which
seems to be common for all cases in which the weight on quality is 0.7
regardless of the other parameters. The adaptive model only partially reproduces
this behavior but the factionalization result is not sustain forever. 

\begin{figure}[!h]
  \centering
  \caption{Evolution of ideology}
  \includegraphics[width = 14cm]{../img/ideology_ruler_midhigh_midhigh.png}
  \label{fig:quality_midhigh_midhigh}
\end{figure}

\begin{figure}[!h]
  \centering
  \caption{Evolution of quality}
  \includegraphics[width = 14cm]{../img/quality_ruler_midhigh_midhigh.png}
  \label{fig:quality_midhigh_midhigh}
\end{figure}

\clearpage

%%% Local Variables:
%%% mode: latex
%%% TeX-master: "promotions_paper"
%%% End:


\section{Conclusions}
\label{sec:conclusions}

In this paper I have explored the consequences of different promotion systems
using a computational model. The trade-off between a competent and a sympathetic
military is explicitly model by allowing an external ruler to choose the
composition of its high command.


%%% Local Variables:
%%% mode: latex
%%% TeX-master: "promotions_paper"
%%% End:

%%%%%%%%%%%%%%%%%%%%%%%%%%%%%%%%%%%%%%%%%%%%%%%%%%%%%%

\newpage

\bibliographystyle{apalike}
\bibliography{/Users/gonzalorivero/Documents/bib/ccss}

\end{document}




%%% Local Variables:
%%% mode: latex
%%% TeX-master: t
%%% End:
