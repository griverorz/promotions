\section{Theory}

Civil-military relations have received considerable attention in the specialized
literature \citep{acemoglu:2010,besley:2010,leon:2009}, as a result of the
interest in the mechanisms that guarantee civilian supremacy. This interest has
natural implications for the study of institutions under autocracy
\citep{svolik:2012} as military obedience is behind the ability of the ruler to
repress his opposition ---which speaks directly to the literature on regime
survival.

Most of the research on the topic of civil-military relations builds on the
notion of a military that behaves as a unitary, congruent institution. With that
constraint, the problem can be formulated as one of a principal-agent relation
in which the agent (the military) exploits its unique expertise (the use of
force) to extract rents of some kind from the principal (the ruler). But the
simplifying assumption of corporate behavior comes at the price of leaving out
the fact that divided militaries are the rule rather than the exception in
historical studies
\citep[see][]{tiruneh:1993,mailer:2012,devlin:1976,sanchez:2002,acosta:2007,smallman:2002}.
It is not difficult to trace evidence of factionalization within autocratic
militaries that manifest either informally as communities of shared interests or
in a more formal ways, casted into a variety of secret (secret military
societies, military networks of political parties) and public (military clubs
and associations) organizations
\citep{smallman:2002,potash:1969,potash:1980,potash:1994}.

The literature has paid attention recently to the divisions that cross the
military and how these conflicts affect the ability of the government to direct
the decisions and actions of the military \citep{tyson:2012, rivero:2011c,
  eynde:2011} but much less attention has been devoted to the causes of internal
fragmentation. In most literature, factions are either taken as a given without
explaining their origin \citep{rivero:2011c} or they are assumed away by
focusing on a military that is composed of individual generals
\citep{tyson:2012}.

But the obvious question is then how it is possible for the military, an
institution that depends on internal cohesion, to generate and sustain these
internal divisions. The qualitative historical literature poses some hypothesis,
mostly related to the promotion systems \citep{huntington:1957}. At the end of
the day, it is in the interest of every ruler to manipulate the composition of
the military in its own benefit in order to count with supporters among the high
command. However, the theory is underdeveloped, probably because any statement
about a promotion system will be very difficult to translate into the empirical
record. 

In addition, at a theoretical level, the strategy for the ruler seems obvious to
delve further into it: the ruler is strictly better off by rewarding and
appointing supporters to the general staff, while punishing or removing
detractors. Naturally, the two obvious can be used in different mixes.

In an extreme case, we would observe the situation illustrated by the Stalinite
army during the Great Purge of 1936--1938 and the Purge of 1941, with the
systematic elimination of the officers of dubious loyalty to the Communist
Party. But such a strategy carries an obvious side effect in the decrease of the
general quality of the office corps, as demonstrated in the inability of the Red
Army to successfully repel the Nazi ground invasion and the inordinate number of
casualties of the Soviet Union during World War II. In addition, purges are
likely to reduce the attractiveness of command positions for the most qualified
soldiers, which implies a further reduction in the overall quality of the
institution via the supply side. Therefore, the ruler must be willing to accept
a military of a lower quality if yet politically devoted.

In the other extreme, the ruler may try to co-opt the military without replacing
officers in the hierarchy as a way to retain high quality officers that disagree
with the ruler. Corporate bribes in the form of high military spending
\citep{powell:2009b} or more targeted benefits such as direct bribes
\citep{goda:2000} or selective promotions are very common and easily found in a
literature that provides all sorts of examples through which the ruler can pour
money into the coercive apparatus. In this way, the ruler can still exert
influence without having to disrupt the internal organization of the military.
Therefore, the ruler can still take advantage of a promotion system strictly
based on quality, at the price of a loyalty that depends on producing a
sufficient flow of resources.

These two cases illustrate the trade-off that the ruler faces but of course the
array of strategic uses of promotions can go much further and have other
---complementary--- motivations. The case of General Franco in Spain
(1939--1975) is a good example as it illustrates the wide menu of alternatives
and strategies that are available to the ruler, specially if he has freedom of
action in the military institution \citep{baquer:2005}. For instance, Franco is
known to have promoted one of his main rivals to a newly created rank of Captain
General, senior to any other general other than Franco himself, with the sole
purpose of removing him from positions with direct command over troops in case
he ever tried to challenge him \citep{baquer:2005}. He also packed the lower
echelons of the chain of command with loyal supporters close to the Falangist
party (the \emph{alf\'ereces provisionales}) early on in his rule as a way to
ensure a supply of mid- and high-ranking officers in the long run
\citep{busquets:1993}. He was also careful to promote to top positions
individuals from opposing factions (conservatives, monarchists or falangists)
with the aim of neutralizing threats from the high command trying to balance out
each political family \citep{busquets:1993}. In some other cases, and in spite
of Franco's devotion for procedures and norms, he was known to overlook question
marks in the profile of candidates for very senior positions, as in the case of
the promotion of Admiral Carrero Blanco, his Prime Minister, who was known to
have skipped some of the mandatory instruction \citep{medina:2004}.

Some other cases are more blatant and include the promotion, not always
successful, of supporters of the coup to high ranks or exceptionally odd
examples such as the promotion of the civilians members of the autocrat's
cabinet to field rank positions. However, such extreme violations of the rules
are uncommon and usually lead to violent reactions from the rest of the
military. As a matter of fact, manipulations of the promotion system are not
always welcome and complains over individual promotions or the promotion rules
as a whole are seen as the motivation for some coups.

% However, manipulations need not be so outrageous. Leaders may still have
% preferences over promotion systems if they are aware of who will be favored by
% them. The clearest example appears again during the Spanish Restoration and the
% several military crises related to the Juntas de Defensa, union-like
% institutions that served as the organic representation of military officers
% resentful about the reforms proposed by the Prime Minister Manuel Garciia Prieto
% regarding the inflated number of officers and the political abuse of promotions
% awarded by war merits. They created a cleavage between generals and junior
% officers, but also between the africanistas (officers fighting in the Moroccan
% War who would benefit from battlefield promotions) and peninsulares (those
% remaining in the Peninsula and in favor of a purely seniority based system). The
% reforms of Primo de Rivera (from the peninsular faction) in 1925 and 1926 would
% be precisely oriented towards favor his own group \citep{boyd:1979}. 

In spite of their presence everywhere, ongoing hierarchical institution remain
one of the least understood objects in political science. Political parties,
administration, companies, the military, \dots{} they all are characterized by a
well defined hierarchy in which some rules exist for moving up through the ranks
into positions with more prestige or power. In some instances, the hierarchical
structure may be argued to pose second order effects for the determination of
the strategy of the institution as a whole in its interaction with other
political agents, but to the extent that the promotion system determines who
arises to the top of the structure and who does not ---hence triggering some
grievances---, it is clear that we need to have a better understanding of the
effects of different promotion rules and the incentives for any decider to
choose among them. 

% swAnd yet, promotion systems in hierarchical institutions remain barely
% understood. Some computation literature on the Peter effect and optimal pay
% scales in Economics are are not entirely well understood. Some literature in
% economics about the optimal pay scale. Some literature in computational social
% science and organizations about promoting based on quality and odd laws. In any
% case, no much literature about the military.

% This paper also relates to the literature on organizations and political
% parties. 

% The autocrat does not have the full autonomy or even the capacity to decide over
% all the appointments for each and every rank or office. The military is a very
% large institution and even the most distrutful ruler must delegate the
% appointments of non-senior officers. In addition, rulers typically have to abide
% by rules that ensure the correct alignment of incentives in a hierarchical
% institution, promotions are expected to follow some kind of seniority rule. He
% may get away with ad hoc appointments or biased promotions, but it is unlikely
% that he could systematically transform civilians into generals without
% disgruntling the officers in line for that position ---although such extreme
% cases are not unheard off.

%%% Local Variables:
%%% mode: latex
%%% TeX-master: "promotions_paper"
%%% End:
